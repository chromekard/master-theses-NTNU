%% This document gives an example on how to use the ntnumasterthesis
%% LaTeX document class.

%% Use short name MACS, MIS, CIMET, MTDMT, MIXD or MIS  
%% Language english or norsk
%% b5paper with oneside or twoside, you can set A4 if you want but you submit in b5

%% If you want print with the heading material on a4 paper you can use this format
%% \documentclass[MACS,english,a4paper,oneside,12pt]{ntnuthesis/ntnuthesis}

%% with the change to using DAIM we have a new option. include DAIM after english below removes the front page material so that you can then submit in the DAIM system. If you are wanting the front material remove DAIM and make sure you fill in the DaimData.tex file.
\documentclass[MACS,english,DAIM]{ntnuthesis/ntnuthesis}

\usepackage[T1]{fontenc}
\usepackage[utf8]{inputenc}     % For utf8 encoded .tex files allows norwegian characters in the files. This can be dangerous if you change to a differnt editor.
%\usepackage[pdftex]{graphicx, hyperref}   % For cross references in pdf
\usepackage{graphicx}
\usepackage{hyperref}   % For cross references in pdf
\usepackage{multirow}
\usepackage{graphicx}


\usepackage{color}              % For colouring text 
\hypersetup{colorlinks=true,     
		linkcolor=blue,          % color of internal links (change box color with linkbordercolor)
    citecolor=blue,        % color of links to bibliography
    filecolor=blue,      % color of file links
    urlcolor=blue           % color of external links
		}
\usepackage{csvsimple}  % for simple table reading and display
\usepackage{url}
\usepackage{booktabs}
\usepackage{gnuplottex} %miktex option if using miktex on windows


\definecolor{darkgreen}{rgb}{0,0.5,0}
\definecolor{darkred}{rgb}{0.5,0.0,0}

\lstset{        basicstyle=\ttfamily,
                keywordstyle=\color{blue}\ttfamily,
                stringstyle=\color{darkred}\ttfamily,
                commentstyle=\color{darkgreen}\ttfamily,
}


%Typesetting of C++ but not always stable in titles etc...
\newcommand{\CPP}[0]{{C\nolinebreak[4]\hspace{-.1em}\raisebox{.1ex}{\small\bf +\hspace{-.1em}+\ }}}

\newcommand{\com}[1]{{\color{red}#1}} % supervisor comment
%\renewcommand{\com}[1]{} %remove starting % to remove supervisor comments
% This will appear in text \com{Lecuters comment} and be visible unless you uncomment
% the renewcommand line.

\newcommand{\todo}[1]{{\color{green}#1}} % items to do
%\renewcommand{\todo}[1]{} %remove starting % to remove items to do

\newcommand{\n}[1]{{\color{blue}#1}} % other comment
%\renewcommand{\n}[1]{} %remove starting % to remove notes

\newcommand{\dn}[1]{} % add the d to a note to say that you have finished with it.





% Set to true ONLY if using Harvard citation style
\newboolean{HarvardCitations}
\setboolean{HarvardCitations}{false} % false for computer science, true for interaction design and harvard style


\ifthenelse{\boolean{HarvardCitations}}{%
	\usepackage{natbib} % for Harvard names as citations.
}{%
	\usepackage[numbers]{natbib} % for Vancover numbers in bibliography
}

\newcommand{\q}[1]{\leavevmode\marginpar{\small\em #1}}
\renewcommand{\q}[1]{}


\begin{document}

% for students submitting in the DAIM system this information will not be used.
% their is an option for DAIM submission which removes this information and checks it is B5.
% Removing the DAIM option on the document type will use this material.

\setthesistitle{Example Masters Thesis. With a long title to test the wrapping of the box}
\setthesisshorttitle{Example Masters Thesis} % a short version for the page headers if your normal title is too long to fit
\setthesisauthor{Bjorn Nodland Fuglestad}
\setthesissupervisor{Simon McCallum}
\setthesissupervisorA{}  % if you have a second supervisor add it like this
%\setthesissupervisorB{Prof. Smart Guy}  % if you have a second supervisor add it like this


\nmtkeywords{Thesis, Latex, Template, IMT}
%\nmtdesc{This is the short description of a masters thesis}


\setthesisdate{01-06-2018}
\setthesisyear{2018}



%for CIMET theses you need to see all of these as well

%\setthesiscampus{Gj\o{}vik}
%\setthesisHostInstitution{\NTNU}
%\setthesisHostInstitution{University of Eastern Finland}
%\setthesisHostInstitution{Universit\'e Jean Monnet Saint-Etienne}

%\setthesisjuryA{} %jury names
%\setthesisjuryB{} %jury names
%\setthesisjuryC{} %jury names
%\setthesisjuryD{} %jury names


 % this is the file which contains all the details about your thesis
\makefrontpages % make the frontpages
%this is the intro to the thesis
\include{preface}


\include{acknowledgments}
The author would like to thank all the volunteers for having the time to participate in the experiment, and would like to extend their gratitude to their supervisor for the support on this publication.

\include{Abstract}
Redirected Walking uses rotation, translation and curvature gains to manipulate users of VR environments. This paper presents a new detection threshold for rotation, translation and curvature gain for the HTC Vive. The calculated rotation gain was 13\% increased and 21\% decreased rotation. For translation gain, 19\% increased and 5\% decreased movement. The detection threshold for curvature gain was a circle with a radius of less than 29m. When comparing these results to previous studies, which used hardware with lower resolution and FOV, the results indicate that it is easier to detect two type of gain, increased rotation and decreased translation in the HTC Vive. Further we discuss if there is a range beyond detection, but that is still comfortable for the user. The results indicate that for some, there are detectable gains that are still comfortable, but this is extremely dependant on the person's sensitivity to the given gain.


\tableofcontents

\hypersetup{pageanchor=true}

% Comment with a percent to remove figures or tables:
\listoffigures
\listoftables

% Chapters

\chapter{Introduction}
\label{chap:introduction}

% - For many of you, this Master's thesis will be the most advanced academic document you ever write.  It needs to demonstrate both academic ability and clear thinking. You Master's should show that you are ready to lead other people, reflect more deeply, and have a professional attitude to your work and environment. 

% When writing the thesis it is important to know who you are writing for. The target audience for this document is in layers:
% \begin{enumerate}
%     \item The marking committee
%     \item Your supervisor
%     \item Other students at the same level 
%     \item Professionals \& Academics
%     \item The general public.
% \end{enumerate}


%Background
% - A short summary of background and previous work.
% - Typically citing the most important and relevant papers of the field
% - Should lead up to the claim

With virtual reality headsets becoming a more and more available to the general public through systems like the Oculus Rift, HTC Vive, Playstation VR and Samsung gear. With the more heavy powered VR systems like HTC Vive and Oculus rift supporting room scale tracking a subject that has received a lot of studies is designing locomotion system to explore the virtual world outside the bounds of the tracked space while remaining inside the physical tracked space. With several of the devices, the featuring position tracking several different locomotion methods have been researched. Based on research of \cite{peck2012design}, \cite{ruddle2009benefits}, \cite{usoh1999walking} and \cite{nabiyouni2015comparing} the results indicate that natural walking and by extension redirected walking provides an intuitive, natural and immersing way to do locomotion in VR over several other locomotion methods. As such redirected walking is an important locomotion method with the possibility of exploring outside the tracked space. Redirected walking was developed by Sharif Razzaque in 2005\cite{razzaque2005redirected} and features a number of techniques used to transform the position of the tracked space in the virtual world based on the user's movement in the physical world. At the heart of redirected walking techniques such as steer to center and steer to orbit are gains. Gains are used to create a difference between the physical transform and the virtual transform. The most common gains used are Rotation, translation and curvature gain. With rotation gain, the rotation made in the physical world is reduced or increased in the virtual world. As with translation gain, the distance moved in the physical world is increased or decreased in the virtual world. With curvature, a straight path is turned into a bent path. Using the differences between virtual and physical position caused by the gains it becomes possible to explore the virtual space beyond the initial tracked space. However, the major weakness of redirected walking is the large area that is required for it to function without interruption. One of the reasons for this is because gains have a threshold for how large the gain can be without being noticed by the user. Several studies have made measurements of the detection threshold the most cited being "Estimation of detection thresholds for redirected walking techniques"~\cite{steinicke2010estimation} from 2010, but was done with what would now be considered outdated equipment. Since then several studies have also measured the detection thresholds with more modern equipment~\cite{meyer2016detection}, however most of them have had additional test parameters which makes it difficult or even impossible to measure the effects of hardware changes. Further not all the detection thresholds have been measured with the latest equipment such a translation gain.
\bigskip

%Claim
% - The most important part of the introduction.
% - Explain why your contribution is important.
% - If you like using personal pronouns like ‘I’ and ‘we’, this is a good place 
% - Should appear as a consequence of the review and lead up to the agenda

\noindent
With this in mind, the purpose of this report is to serve as a trail for my master thesis which intends to calculate the threshold values for rotation, translation and curvature gain with the latest commercial equipment. By recalculating the detection threshold it will be possible to compare the results with older studies and evaluate the effects of the hardware changes on the detection thresholds. In addition to looking into the thresholds for detection, I also intend to look into the maximum thresholds where the gains are detectable but still usable. The reason for looking for this threshold is that it has value for dynamic redirection techniques that apply higher redirection the further from the center that the user is. For such techniques knowing the max gains that can be applied allows the technique to be exploited to its full potential. As this is a trial run only the detection and maximum thresholds for rotation gain will be implemented. The reason for using rotation gain is that it is the gain that requires the least amount of space to test. In addition, I will also be taking this opportunity to get user feedback for improving the implementation of the experiments used to get the thresholds.
\bigskip
% https://docs.unity3d.com/Manual/OpenSourceRepositories.html

%Agenda (in my case optional)
% - A short summary of the contents of the report and how it fulfills the claim.
% - Should not be the table of contents in paragraphed form (it often is) 
% - Also this can be a good place for personal pronouns

\noindent
As such the report will explain how the rotation gain is implemented and the setup for the experiments needed to calculate the threshold values. The discussion will also be discussed along with lessons learned during the implementation and what changes need to be made for the master thesis implementation. % includes latex files from the same directory
\chapter{Related work}
\label{chap:related_work}
\chapter{Methods}
\label{chap:methods}

The \texttt{ntnuthesis} is built upon the standard \LaTeX\
\texttt{report} class. All commands from the \texttt{report} class can
be used, with the two exceptions of \verb+\subsubsection+ and
\verb+\paragraph+. This is because there should only be three
levels of headings according to the guidelines. 
It has been placed in a folder called \texttt{ntnuthesis} so that it does not
clutter your work.  You should not change anything in \texttt{ntnuthesis}. If you need to change 
anything you should make a pull request on the github repository for this thesis at
\url{https://github.com/COPCSE-NTNU/master-theses-NTNU}

\section{Packages Used by ntnuthesis}
\label{sec:packages}

In addition to the \texttt{report} document class,
\texttt{ntnuthesis} makes direct use of the following packages
that must hence be present:
\begin{description}
	\item[geometry:] used for setting the sizes of the margins and
  	headers.
	\item[fontenc:] used with option \texttt{T1} for forcing the Cork font
  	encoding (necessary for the Charter font).
	\item[charter:] load Charter as the default font.
	\item[euler:] load the Euler math fonts.
	\item[bable:] for language handling.
	\item[listing:] for code listing.
\end{description}

\section{Other Relevant Packages}
\label{sec:otherpackages}

The author of a thesis might want to use a bunch of different packages
to those described in Section~\ref{sec:packages} in order to have all features needed for their document. 
In particular, it is advised to use the following:
\begin{description}
	\item[inputenc:] to allow \LaTeX\ to use more than 7-bit ASCII for its
	  input. Most often, the option \texttt{latin1} will do.
	\item[babel:] to load language specific strings. Reasonable options
	  include \texttt{british}, 
		\texttt{american}, \texttt{norsk} and
	  \texttt{nynorsk}.
	\item[graphicx:] to include graphics.
	\item[hyperref:] this is a very nice package that makes cross links in
	  pdf documents. Use with option \texttt{dvips} or \texttt{pdftex}
	  in accordance with the driver that you use. Unfortunately, hyperref
	  is not completely bugfree\dots
\end{description}

We have web pages as well~\cite{NTNU:Website}, and now games like Halo~\cite{Halo}. % could be called Methodology or methods or any filename
\chapter{Implementation}
\label{chap:implementation}
% - This has the description of how you actually went about implementing the project.  This should be focused on the interesting challenges and how those related to the project.

The software used for measuring the detection thresholds was made in unreal engine 4 version 4.18.3. The virtual environment in which the experiments were tested was built to look like a skyscraper with each of the different test taking place in different rooms, as seen in figure~\ref{fig:game_room}. All the assets used were either from epic games blueprint tutorial, unreal engine's starter content, VR package, first-person shooter package or made by the author. Each of the floors was decorated sufficiently to provide enough visual density \cite{paludan2016disguising} for redirected walking to be noticeable.

The implementation of redirected walking for this paper took inspiration from the redirected walking toolkit \cite{azmandian2016redirected}. However, given the different game engines used the final product became quite different. The object used for implementing the gains in this paper featured a root component, a camera component, two motion controller components with a controller static mesh subcomponent for each of the motion controller components. As the camera component is controlled through the HMD it's relative position and rotation can be used to monitor the movement of the wearer. To be able to apply the different gains the delta rotation and delta movement of the camera component was calculated at each tick. The delta rotation is calculated as the signed angle changes along the z-axis (yaw), while the delta movement is calculated as the difference between current and previous position but only along the x and y-axis. To avoid affecting the relationship between the relative position of the camera and the real position of the VR headset the gains were applied to the whole object instead of the camera component.

\begin{figure}
    \centering
    \includegraphics[width=\linewidth]{figures/InGameView.png}
    \caption{VR room used for translation and curvature in experiment 2}
    \label{fig:game_room}
\end{figure}

\subsection{Rotation gain}
The initial step for applying rotation gain is calculating how much the virtual rotation of the actor needs to change ($\Delta R_{virtual}$) to apply given the changes in the real rotation ($\Delta R_{real}$) and the rotation gain ($G_{r}$)
\[\Delta R_{virtual} = \Delta R_{real} * G_{r}\]
For rotation gain, it was required to smooth the rotation as low rotation gain values caused jittering. To calculate the smoothed rotation ($sR_{virtual}$) the last rotation ($lR_{virtual}$) applied was used along with a smoothing factor ($S_f$) of 0.5.
\[\Delta sR_{virtual} = (R_{virtual} * S_f) + (lR_{virtual} * (1 - S_f)) \]
Having found the rotation the next step is to find how much the object needs to be moved ($\Delta Pos $) so that the rotation applied to the object will cause the camera object to be rotated while staying in the same position. To find this position we first get the vector ($V_{CO}$) from the camera (C) to the object (O).
\[V_{CO} = O_{pos} - C_{pos} \]
Then the vector is rotated around the z-axis. We then subtract the rotated vector by the vector from the camera to object again as this will give us the vector for how much the actor is required to move, as the rotated vector only gives how much the camera would need to move to reach the same position.
\[\Delta Pos = RotateAroundAxis( V_{CO}, sR_{virtual}, Z_{axis}) -  V_{CO})\]
The $\Delta Pos$ and $\Delta sR_{virtual}$ are then added to the objects world transform. 

\subsection{Translation gain}
Before applying the translation gain ($G_{t}$) the changes in real movement ($\Delta Mov_{real}$) needed to be rotated to match the objects rotation. If not the translation gain would not be applied in the right direction while the objects rotation was not 0 around the z axis.
\[r\Delta Mov = RotateAroundAxis(\Delta Mov_{real}, O_{rot.z}, Z_{axis}) \]
Then how much the object should be moved can be calculated.
\[\Delta Mov = Normalize(r\Delta Mov) * ( |r\Delta Mov| * G_{t})\]
The $\Delta Mov$ is then added to the actors world offset.

\subsection{Curvature gain}
Curvature gain ($G_{c}$) is implemented similarly to rotation gain with the exception being how the change in virtual rotation ($\Delta R_{virtual}$) is calculated.
\[\Delta R_{virtual} = (\frac{|\Delta Mov_{real}|}{G_{c}}) * 360\]
Then as with rotation gain, the same math is used to calculate what position that needs objects to be moved to for the rotation to apply to the camera but not move it. The rotation and distance are then added to the objects world transform. In order to avoid dividing by 0 or any other errors related to 0 values if statements checked that the gain, delta movement or delta rotation not 0 before calculated and applying gains.
\chapter{Results}
\label{chap:results}

If you are submitting using the DAIM system you should make sure the pdf you submit does not have the front page information, as that will be added by the submission form in DAIM.  You can remove the DAIM option to print the front page material if you want a full PDF with the front page material. To make sure the running header has the title of the thesis you still need to set it in the \verb+DaimData.tex+ file. The title of the thesis should be set using the \verb+\thesistitle+
command, and the date of the thesis should be set using the
\verb+\thesisdate+ command in the \verb+DaimData.tex+ file. 

\section{Page Layout}

The geometry of the page has been set using the \verb+\geometry+
command.

\section{Fonts}

Due to limited \LaTeX\ support for the Georgia font, Charter has been
chosen instead. For mathematical formula, the Euler fonts are used,
since they blend more nicely with the Charter than the standard
\LaTeX\ fonts: 
\begin{equation} \label{eq:1}
    f(x) = \int_0^x g(\tau)\,d\tau
\end{equation}



For inline math you can use $\backslash{}($ and $\backslash{})$ for example \( f(x)= \frac{x^2}{1+x^2} \).  
This also allows you to use $\slash$ and $\backslash$. You need to include the \{\} when you want the special
character to have other letters immediately after it.

\section{Sectioning Commands}

The standard \LaTeX\ sectioning commands are used for both numbered
and unnumbered sections. The top level is given by the \verb+\chapter+
command. This starts a new right page. The two lower levels are
obtained using the \verb+\section+ and \verb+\subsection+ commands.
The standard \LaTeX\ \verb+\subsubsection+ and \verb+\paragraph+
commands have been disabled since their use is not encouraged by the
thesis guidelines. When you use these they will not be given numbers.  
They still appear in the document with highlighting but not in the 
table of contents.

\subsection{The subsection}

This is an example of a subsection.

\subsubsection{The subsubsection}

This is an example of a subsubsection.

\paragraph{The paragraph}

This is an example of a paragraph with a heading.

\section{Floats (Figures and Tables)}
\label{sec:floats}

Figures are placed in the \texttt{figure} environment. An example is
shown in Figure~\ref{fig:example}. %notice the ~ in between figure and the \ref. it stops latex from splitting the number and word over a line.
Tables are placed in the \texttt{table} environment. An example is given in
Table~\ref{tab:example} and reading the information directly from file in Table~\ref{tab:examplecsv}. Figures and tables float freely around in the
document in accordance with standard \LaTeX\ behavior.

\begin{figure}[tbp]  %t top, b bottom, p page | you can also use h to try to get the figure to appear at the current location
  \centering
  \includegraphics[width=.5\textwidth]{figures/example_fig}
  \caption[An example figure.]{An example figure. If the caption is
    shorter than one line, it is centered. If it goes over more than
    one line, it is left and right justified. Furthermore, it is
    suggested that an alternative short caption is given in order to
    produce a good list of figures.}
  \label{fig:example}
\end{figure}

\begin{table}[tbp]
  \centering
  \begin{tabular}{c|c}
    Age  & IQ  \\ 
    \hline
    10   & 100 \\
    20   & 100 \\
    30   & 150 \\
    40   & 100 \\
    50   & 100
  \end{tabular}
  \caption{An example table.}
  \label{tab:example}
\end{table}

\begin{table}[tbp]
  \centering
  \csvautobooktabular{figures/ageiq.csv}
  \caption{An example table using simplecsv.}
  \label{tab:examplecsv}
\end{table}

The captions are placed \emph{below} both for the figures and the
tables. The caption is set in 9pt. If the caption is shorter than one
line, it is centered.

\subsection{Gnuplot}
There are many ways to include graphs in your document.  Figure~\ref{fig:exgnuplotex} for including a file generated by gnuplot and saved as \texttt{gnuplotgraph1.tex}. 
%Figure~\ref{fig:exgnuplotintegrate} shows how to include the script to generate a graph direction in \LaTeX.

\begin{figure}[htp]  %t top, b bottom, p page | you can also use h to try to get the figure to appear at the current location
  \centering
  \input{figures/gnuplotgraph1}
  \caption[An example graph.]{This is a gnuplot graph of $y=\sin(x)$. Notice how the \LaTeX{} fonts are preserved in the graph. This is done using gnuplot and the simple text file included in the sample template.}
  \label{fig:exgnuplotex}
\end{figure}

\begin{figure}[htp]  %t top, b bottom, p page | you can also use h to try to get the figure to appear at the current location
  \centering
    \begin{gnuplot}[terminal=epslatex, terminaloptions=color]
        set xlabel "Age" 
        set ylabel "IQ" 
        set key autotitle columnhead
        set title "Age vs Average IQ"
        set yrange [0:160]
        set datafile separator ","
        plot "figures/ageiq.csv" using 1:2 with boxes 
    \end{gnuplot}
  \caption[An example of Integrated Graph]{This is a gnuplot graph read from a file}
  \label{fig:exgnuplotintegratefile}
\end{figure}


%\begin{figure}[htp]  %t top, b bottom, p page | you can also use h to try to get the figure to appear at the urrent location
%  \centering
%    \begin{gnuplot}[terminal=pdf, terminaloptions=color]
%        unset hidden3d
%        set view 102,57,1
%        set xtics offset -1.3,-0.3
%        set ytics offset 0,-0.5
%        set samples 21
%        set isosample 11
%        set xlabel "Confidence" offset -3,-2
%        set ylabel "Resilience" offset 3,-2
%        set zlabel "Rate of change" offset 2, 6
%        set title "Rate of feat change in relation to Resilience and Confidence"
%        set xrange [0:1]
%        set yrange [0:1]
%        splot 1-((1-x)*y)
%    \end{gnuplot}
%  \caption[An example 3D graph.]{This is a gnuplot graph of $1-((1-x)*y)$. This is code that is compiles during the \LaTeX{} processing. This is done using gnuplottex, it could also come from a file}
%  \label{fig:exgnuplotintegrate}
%\end{figure}

\section{Quotes}
\label{sec:Quotes} % this allows you to refer to this section number using \ref{sec:Quotes}

Quotes are inserted using the standard \LaTeX\ \texttt{quote}
environment. The environment has been changed so that a 9pt font is
used:

\begin{quote}
  ``And I looked, and, behold, a whirlwind came out of the north, a
  great cloud, and a fire infolding itself, and a brightness was about
  it, and out of the midst thereof as the colour of amber, out of the
  midst of the fire. Also out of the midst thereof came the likeness
  of four living creatures.''
\end{quote}

\section{Lists}
\label{sec:lists}

Point lists and enumerated lists are made by using the standard
\texttt{itemize} and \texttt{enumerate} environments, respectively.
The spacing is going to be changed in accordance with the specification. For
\texttt{itemize}, the results look like this:
\begin{itemize}
	\item First item.
	\item Second item. Here I will put some long text, just to illustrate.
	  Here I will put some long text, just to illustrate. Here I will put
	  some long text, just to illustrate. Here I will put some long text,
	  just to illustrate.
	\item Third item also has subitems:
	  \begin{itemize}
		  \item First subitem.
		  \item Second subitem.
		  \item Third subitem.
	  \end{itemize}
\end{itemize}
and for \texttt{enumerate} like this:
\begin{enumerate}
	\item First item.
	\item Second item. Here I will put some long text, just to illustrate.
	  Here I will put some long text, just to illustrate. Here I will put
	  some long text, just to illustrate. Here I will put some long text,
	  just to illustrate.
	\item Third item also has subitems:
	  \begin{enumerate}
		  \item First subitem.
		  \item Second subitem.
		  \item Third subitem.
	  \end{enumerate}
\end{enumerate}

You may also want to use descriptive lists
\begin{description}
	\item[First] the first item.
	\item[Second] the second item. Here I will put some long text, just to illustrate.
	  Here I will put some long text, just to illustrate. Here I will put
	  some long text, just to illustrate. Here I will put some long text,
	  just to illustrate.
	\item [What now] the third item also has subitems:
	  \begin{enumerate}
		  \item First subitem.
		  \item Second subitem.
		  \item Third subitem.
	  \end{enumerate}
\end{description}


\section{Bibliographic References}

There are two distinct styles of referencing which can be used within the Masters thesis, Vancouver for Computer Science and Harvard for Interaction Design.

In Computer Science we generally use the Vancouver style with numbered references.  
I have added a boolean option \verb|\setboolean{HarvardCitations}{false}|  Havard style if false for computer science and true for interaction design.
 
In the Vancover style you should cite articles~\cite{Askvall1985}, books~\cite{Card1983},
anthologies~\cite{Lancaster1985} and web publications~\cite{Meldon1997}
like this. For all citations note that in the text there is the tilde \~ character.  
That is a non-breaking space which forces the number to stay with the text rather than move to the next line.
There is always an issue referencing web pages. Currently
we suggest that you use the NTNU Website~\cite{NTNU:Website}.


For Harvard style referencing, you use the \texttt{citep} and \texttt{citet} style of citation. 
These give parentheses around the citation or the name of the author as text with the year in parentheses.  
If you want the citation to be read in a sentence then you use  \texttt{citet}. 
If you want it to be just parenthetical to the sentence at the end, then use \texttt{citep}.

\section{Code}

For code listing (see Figure~\ref{fig:HelloWorldC++} and Figure~\ref{fig:PythonCode}) we have included the listings package so that you can easily include formatted code.  It does not have code highlighting but it retains the structure of the code.  For more documentation on listings on wikibooks \footnote{\url{https://en.wikibooks.org/wiki/LaTeX/Source_Code_Listings}}


\begin{figure}[tp] 
  \centering
\lstset{language=C++,
        morecomment=[l][\color{darkgreen}]{\#}}
\begin{lstlisting}
    #include<stdio.h>
    #include<iostream>
    // A comment
    int main(void)
    {
    printf("Hello World\n");
    return 0;
    }
\end{lstlisting}
  \caption[Hello World C++]{The code listing for Hello World in C++, with colour syntax highlighting.}
  \label{fig:HelloWorldC++}
\end{figure}

You could also use Python code listings by changing the language of the code block

\begin{figure}[tp] 
  \centering
\lstset{language=Python}
\begin{lstlisting}
import numpy as np
x = 1
a = np.array([[1.0, 2.0], [3.0, 4.0]])
if x == 1:
    # indented four spaces
    print("x is 1.")
    print("Hello World")
    print(a)
\end{lstlisting}
  \caption[Python code example]{The code listing for a Python increment a matrix example}
  \label{fig:PythonCode}
\end{figure}

\section{Statistical Analysis}

Many of you will need to use statistics to reject null hypotheses.  There are many statistical packages and ways of analysing data.  Your supervisor should be able to direct you to the type of analytically tool that will allow you to make justifiable claims.

There are some key things to remember.  If you want to make a claim that thing A is better than thing B, then you are rejecting the null hypothesis that they are the same. Equation~\ref{H0mean} states the null hypothesis as the mean of sample 1 ($\mu_1$) is the same as the mean of sample 2 ($\mu_2$). For example if you were measuring height of men and women you would state that the null hypothesis is that men and women are the same height, then you measure 100 men and 100 women and calculate the mean and standard variation of their height. 
\begin{equation} 
\label{H0mean}
    H_0 : \mu_1 = \mu_2
\end{equation}

The t-test can be used to see what the probability $p$ of seeing the values in the sample that you coming from the same actual population. If you have a $p<0.05$ you have a 95\% probability that the samples are actually different.

Thus you set up the evaluation and show that there is a very low probability that the difference you see is caused by sampling error and therefor they are not the same.  The t-test does this for normally distributed scalar values of data. If you are using a Likert Scale then you do not have scalar data, and it may not be normally distributed.

For non-parametric data you need to make a statement about the sampled values~\cite{Kaptein2010}
\begin{equation} 
\label{H0sample}
    H_0 : \phi_1(x) = \phi_2(x)
\end{equation}

There are lots of good sources for understanding statistics for research.  Most of the wikipedia pages are a good entry to the area. For Likert scale analysis there are new tools~\cite{Kaptein2010} which allow for better assessment of the sample sizes we have in most of our Masters thesis projects.

You should also think about learning R the statisical package for doing analysis.  You can download it by searching for "R on windows" or using the link to a windows implementation of R\footnote{\url{https://cran.r-project.org/bin/windows/base/}}


 % could be results
\include{inc/discussion}
\chapter{Conclusion}
\label{chap:conclusion}

% Content
% - Summaries your main findings; to what degree were your claims fulfilled
% - The conclusion must be supported by the contribution
% - Discuss briefly what relevant work you see as a natural consequence of your work
% - This is where you provide an overview of the thesis now that it is finished.  What are the critical things that can be learnt from the thesis for the reader.

The measured detection thresholds for rotation gain with the latest equipment (HTC Vive) shows that the detection thresholds are when the gain is increased 11\% and when decreased 21\% with a PSE of 0.95. Based on these results and comparing the results with previous studies the results indicate that the detection threshold for rotation gain is affected by changes in the HMD. The changes in hardware specifically reduce how much the rotation gain can be increased before being detected. However, which specific changes to the HMD caused the effect is still uncertain, as it could be both resolution and field of view. As such this will have an effect on how usable redirected walking will be as a VR locomotion method. With redirected waking algorithms using the detection thresholds having a lower range in how much the gain can be increased without being noticed.

When examining the maximum tolerance for rotation gain the results and observations indicate that how much rotation gain a participant can endure very much depends on the person. This was because how much gain the person could endure would depend on their sensitivity to simulation sickness. As such there were some participants that could endure high rotation gains while some who stopped the moment the gains became noticeable. Further, it was also observed that there was a large difference in the amount motion sickness caused by increasing and decreasing the gain. With decreased rotating gain being the most simulation sickness inducing. While high rotation gains had a smaller effect on simulation sickness. As such when using redirected walking as a locomotion method the algorithms in use should to the best possible extent not go over the detection threshold for decreased gains, however, it is possible for them to be less strict on the limit for how much the gain can be increased.

% PAPER VERSION

From the results of this experiment, a new detection threshold was calculated for rotation, translation and curvature gain. The detection thresholds for rotation gain were at 13\% increased rotation and 21\% decreased rotation. When compared to previous studies that look at the threshold for rotation gain the comparison shows that the detection threshold for increased rotation has been reduced on the new hardware compared to the old. However, the detection threshold for decreased rotation has not changed significantly. For translation gain, 19\% increased movement and 5\% decreased movement. The opposite is found when comparing the detection threshold found for translation gain with those of older studies. The comparison shows that how much the movement can be decreased is significantly lower compared to the previous studies. While how much the movement can be increased is about the same. The detection threshold for curvature gain found in this study would cause the user to walk in a circle with a radius of 28.65m. Based on the comparison with older studies this is slightly larger than previous results however not large enough that it can be attributed to hardware instead of other test parameters such as walking speed or the participant bias.

In addition, when looking into if it there was a range of gain values that were noticeable but usable the results indicate that it does exist, however, it is extremely dependant on the person's sensitivity to the given gain. So much so that in some cases it does not exist as the user will become ill the moment they notice that the gain is applied. This sensitivity also applies to when a person first notices a gain. Because of this when users first notice that a gain is applied varies considerably as well. However, on average this point was above the detection thresholds in most cases as such it would still be reasonable to use detection thresholds as the default limits used by redirected walking algorithms. Though the best option would be to have the limits be directly customized to a given user. The usage of this additional range of values would then be used as an additional option for the user if they desired even more redirection. Possibly due to limited space at the cost of immersion.

\section{Future Work}
\label{sec:future}
% - Where would the project go from here.

\todo{again more examples and discussion about what it means to plan}
\todo{there are many more things to say}

As future work finding which specific part of the HMD that causes the changes to the detection thresholds would be an important part of understanding how to counteract the decreased rotation gain detection threshold. Further the effects of the HMD changes have not been tested with all of the different gain methods and the effects on them are currently uncertain.

% Future work paper

While this study has show that hardware specifications have an effect on the detection threshold the question of the what happens in the future still remains. While the HTC Vive is currently considered a modern piece of equipment the second version is currently on it's way. Further other HMD that promise to provide even better specification in terms of resolution and FOV are also being developed such as the PiMax. As such it remains to see if the detection thresholds will change further with even more changes in hardware or if the changes will stagnate. Further in regards to the range of usable but noticeable gains it remains to be seen if can provide a more usable thought less immersing experience in a smaller play area. Such as leading to fewer reset or how much it would effect simulation sickness.

\ifthenelse{\boolean{HarvardCitations}}{%
	\bibliographystyle{agsm} % used for Harvard style references. Names - Humanities & Interaction Design
}{%
	\bibliographystyle{ntnuthesis/ntnuthesis} %used for Vancover style references. Numbers - Computer Science & Physics
}

\bibliography{bib/MastersExample}

\appendix
\chapter{Testing Data}
This could be a log of the testing sessions and raw data that is too long for the thesis.
\section{Test Session 1}
\subsection*{Objective}
The Objective of this testing session was to test Hypothesis ....

\subsection*{Participants}
The participants were a convenience sample of students from NTNU. Age range $19-55$ Median age $23$ ...

\subsection{Raw Data}





%\include{inc/timetable}

\end{document}
