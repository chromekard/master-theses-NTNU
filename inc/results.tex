\chapter{Results}
\label{chap:results}

The participants gathered for the experiment were selected through convenience sampling. The time used per experiment depended on the gain being tested and on the participant. For rotation gain, the time needed ranged from 20 to 40 minutes, with translation gain 40 to 60 minutes with curvature gain 30 to 40 minutes. When fitting the data from experiment 2 the sigmoid function used was the same as in \cite{steinicke2010estimation}.
\[ f(x) = \frac{1}{1 + e^{a*x+b}} \]

\subsection{Rotation gain}
When testing rotation gain a total of 17 participants participated. 14 participants identified as male and 3 as female. The participants were all between 20-30 years of age. On average the participants had large amounts of gaming and VR experience. 7 of the participants used some sort of vision correction and 4 used the vision correction while wearing the HMD. The average SSQ score from the pre-test produced a total score of 10.52 while the post-test gave a score of 27.82 with the changing in the score being 17.30.

\subsubsection{Rotation experiment 1}
Table~\ref{tab:rotation_increment} shows the points at which participants noticed the gain being applied and when the gain became unusable for them for both positive and negative gain. The average detection point for positive translation gain was 0.835 and the average limit was at 1.589 with the difference between the two is 0.753. Opposite with negative translation gain the average detection point was at -0.310, the average limit at -0.449 making the average difference between the two -0.139.

\begin{table}[]
\centering
\begin{tabular}{@{}lllll@{}}
\toprule
UserID & DetectPos & MxGainPos & DetectNeg & MxGainNeg   \\ \midrule
A01    & 0.157      & 0.998     & -0.369     & -0.420    \\
A02    & 0.529      & 1.408     & -0.159     & -0.279    \\
A03    & 0.298      & 0.674     & -0.115     & -0.246    \\
A04    & 1.525      & 3.850     & -0.846     & -0.846    \\
A05    & 0.500      & 0.815     & -0.359     & -0.635    \\
A06    & 0.441      & 0.850     & -0.337     & -0.337    \\
A07    & 1.326      & 2.078     & -0.501     & -0.946    \\
A08    & 0.263      & 0.821     & -0.158     & -0.209    \\
A09    & -          & -         & -0.192     & -0.383    \\
A10    & 0.769      & 3.892     & -0.862     & -0.862    \\
A12    & 1.774      & 1.831     & -0.371     & -0.453    \\
A13    & 3.050      & 3.289     & -0.087     & -0.668    \\
A14    & 0.820      & 0.982     & -0.136     & -0.270    \\
A15    & 0.440      & 1.249     & -0.341     & -0.341    \\
A16    & 0.146      & 0.301     & -0.052     & -0.141    \\
A17    & 0.492      & 0.792     & -0.075     & -0.145    \\ \bottomrule
\end{tabular}
\caption{At which gains the different participants noticed the rotation gain being applied and at which gain they reached their limit.}
\label{tab:rotation_increment}
\end{table}

\subsubsection{Rotation experiment 2}
During this experiment 5 (3F, 2M) of the participants were unable to complete the task because of simulation sickness. The results from fitting sigmoid function, as seen in figure~\ref{fig:rotation_graph}, shows that the PSE is -0.04, with the lower detectable threshold being -0.21 and upper detectable threshold being 0.13. In other words, the virtual rotation can be up-scaled by 13\% and down-scaled by 21\%.

\begin{figure}
    \centering
    \includegraphics[width=\linewidth]{figures/RotationGraph.png}
    \caption{Caption}
    \label{fig:rotation_graph}
\end{figure}

\subsection{Translation gain}
When testing translation gain a total of 11 participants all being able to complete both parts of the experiment. All participants identified as male and were between 20-30 years of age. On average the participants had large amounts of gaming and VR experience. 5 out of the 11 participants used some sort of vision correction and 3 used the vision correction while wearing the HMD. The average SSQ score from the pre-test gave a total score of 4.49 while the post-test gave a score of 6.73. This meant a change in score of 2.24 from pre-test to post-test.

\subsubsection{Translation Experiment 1}
Table~\ref{tab:translation_increment} shows the points at which participants noticed the gain being applied and when the gain became unusable for them for both positive and negative gain. The average detection point for positive translation gain was 0.491 and the average limit was at 1.242 with a difference between the two of 0.751. Opposite with negative translation gain the average detection point was at -0.258, the average limit at -0.473 making the average difference between the two -0.215.

\begin{table}[]
\centering
\begin{tabular}{@{}lllll@{}}
\toprule
UserID & DetectPos  & MxGainPos & DetectNeg  & MxGainNeg \\ \midrule
B01    & 0.503      & 1.194     & -0.236     & -0.883    \\
B02    & 0.659      & 2.141     & -0.580     & -0.855    \\
B03    & 0.144      & 1.180     & -0.245     & -0.525    \\
B04    & 0.705      & 2.583     & -0.400     & -0.543    \\
B05    & 0.177      & 0.351     & -0.403     & -0.403    \\
B06    & 0.366      & 0.491     & -0.092     & -0.279    \\
B07    & 0.900      & 1.737     & -0.091     & -0.326    \\
B08    & 0.383      & 0.829     & -0.274     & -0.325    \\
B09    & 0.464      & 0.812     & -0.097     & -0.255    \\
B10    & 1.003      & 2.086     & -0.286     & -0.616    \\
B11    & 0.098      & 0.253     & -0.138     & -0.196    \\ \bottomrule
\end{tabular}
\caption{At which gains the different participants noticed the translation gain being applied and at which gain they reached their limit.}
\label{tab:translation_increment}
\end{table}

\subsubsection{Translation Experiment 2 }
The results from the fitting the sigmoid function, as seen in figure~\ref{fig:translation_graph}, to the data results in a PSE value of 0.07, with a lower detection threshold of -0.05 and an upper detection threshold of 0.19. This suggests that virtual movement can be scaled down by 5\% and scaled up by 19\%.

\begin{figure}
    \centering
    \includegraphics[width=\linewidth]{figures/TranslationGraph.png}
    \caption{Caption}
    \label{fig:translation_graph}
\end{figure}

\subsection{Curvature gain}
When testing curvature gain a total of 14 participants participated. All participants identified as male and were between 20-30 years of age. On average the participants had large amounts of gaming and VR experience. 6 out of the 14 participants used some sort of vision correction and 4 used the vision correction while wearing the HMD. The average SSQ score from the pre-test gave a total score of 9.88 while the post-test gave a score of 14.96. Making the changes in score 5.08.

\subsubsection{Curvature Experiment 1 }
Table~\ref{tab:curvature_increment} shows the points at which participants first noticed the gain being applied and when the gain became unusable. The average detection point for combined curvature gains was 5269.357 and the average limit was at 3116.569 with a difference between the two of -2152.788.

\begin{table}[]
\centering
\begin{tabular}{@{}lll@{}}
\toprule
UserID & Detected  & MaxGain   \\ \midrule
C01    & 1644.800  & 1644.800  \\
C02    & -5214.817 & -1713.885 \\
C03    & -3761.173 & -3761.173 \\
C05    & 8716.367  & 7200.489  \\
C06    & 11998.981 & 1971.579  \\
C07    & 9429.240  & 1686.915  \\
C08    & -6036.034 & -4624.893 \\
C09    & 9383.335  & 7951.763  \\
C10    & 1422.103  & 1679.129  \\
C12    & 2123.467  & 1662.232  \\
C13    & -1722.557 & -1722.557 \\
C14    & 1779.415  & 1779.415  \\ \bottomrule
\end{tabular}
\caption{At which gains the different participants noticed the curvature gain being applied and at which gain they reached their limit.}
\label{tab:curvature_increment}
\end{table}

\subsubsection{Curvature Experiment 2}
As the method used for finding the detection threshold for curvature gain only looked at one rotation no PSE is measurable. The detection threshold found based on the fitted function was at 17999.32cm, as seen in figure~\ref{fig:curvature_graph}. Which when converted to radius in meters instead of circumference in cm is 28.65m.

\begin{figure}
    \centering
    \includegraphics[width=\linewidth]{figures/CurvatureGraph.png}
    \caption{Caption}
    \label{fig:curvature_graph}
\end{figure}